\documentclass{article}
\usepackage[utf8]{inputenc}
\usepackage[a4paper, portrait, margin=1in]{geometry}
\setlength{\parskip}{1em}


\title{}

\begin{document}

\begin{center}
{\bf PHYS 20323/60323: Fall 2019 - LaTeX Example}
\end{center}

\begin{enumerate}

\item{ Consider a particle confined in a two-dimensional infinite square well

\[V(x,y) = \left\{
  \begin{array}{lc}
    0, & 0 \ge x \ge a, 0 < y < a\\
    {\infty}, & {otherwise}
  \end{array}
\right.
\]

The eigenfunctions have the form:

\begin{equation} 
 \Uppsi(x,y)=\frac{2}{a}sin(\frac{n{\pi}x}{a})sin(\frac{m{\pi}y}{a})
\end{equation}

with the corresponding energies being given by:

\begin{equation} 
 E_{nm}=(n^2+m^2)\frac{{\pi^2}{\hbar^2}}{2ma^2}
\end{equation}

(a) (5 points) What are the levels of degeneracy of the five lowest energy values?

\vspace{3mm}

(b) (5 points) Consider a perturbation given by:

\begin{equation} 
 \hat{H}' = a^2V_0 {\delta}(x-\frac{a}{2}) \delta(y-\frac{a}{2})
\end{equation}

Calculate the first order correction to the ground state energy.}

\vspace{7mm}

\item {{\bf The following questions refer to stars in the Table below}

Note: There may be multiple answers. 

\begin{center}
\begin{tabular}{ | l | c | c | c | c | c |}
 \hline
 Name & Mass & Luminosity &  Lifetime & Temperature & Radius\\ 
 \hline
 Zeta & 60. $M_{sun}$ & $10^6 L_{sun}$ & 8.0 x $10^5$ years & &\\  
 \hline
  Epsilon & 60. $M_{sun}$ & $10^3 L_{sun}$ & & 20,000 K &\\  
 \hline
  Delta & 2.0 $M_{sun}$ & & 5.0 x $10^8$ years & & 2 $R_{sun}$\\  
 \hline
  Beta & 1.3 $M_{sun}$ & 3.5 $L_{sun}$ & & &\\  
 \hline
  Alpha & 1.0 $M_{sun}$ & & & & 1 $R_{sun}$\\  
 \hline
  Gamma & 0.7 $M_{sun}$ & & 4.5 x $10^{10}$ years & 5000 K &\\  
 \hline
\end{tabular}
\end{center}

(a) (4 points) Which of these stars will produce a planetary nebula at the end of their life.

\vspace{10mm}

(b) (4 points) Elements heavier than $Carbon$ will be produced in which stars. }


\end{document}